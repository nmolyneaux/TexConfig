%%%%%%%%%%%%%%%% Packages %%%%%%%%%%%%%%%%
\usepackage{tikz}
\usepackage{pgfplots}
\usepackage{xparse}
\usetikzlibrary{calc}
\usetikzlibrary{arrows}
\usepackage{pgfplotstable}

%% To use a \inputTikz as the \includegraphics works
\newcommand{\inputTikz}[2]{%specify width for tikz images
     \scalebox{#1}{\input{#2}}  
}

%%%%%%%%%%%%%%%% Dimensionning %%%%%%%%%%%%%%%%
%% http://tex.stackexchange.com/questions/14901/dimensioning-of-a-technical-drawing-in-tikz %%
\tikzset{%
    Cote node/.style={%
        midway,
        sloped,
        fill=white,
        inner sep=1.5pt,
        outer sep=2pt,
    },
    Cote arrow/.style={%
        <->,
        >=latex,
        very thin,
    }
}

\makeatletter
\NewDocumentCommand{\Cote}{%
    s       % cotation avec les flèches à l'extérieur
    D<>{(0,0)} % offset of labels
    O{.75cm}    % offset de cotation
    m       % premier point
    m       % second point
    m       % étiquette
    D<>{o}  % () coordonnées -> angle
            % h -> horizontal,
            % v -> vertical
            % o or what ever -> oblique
    %O{1}	% scale for labels 
    O{}     % parametre du tikzset
    %O{0}	% rotation text
    }{%

    {%\tikzset{#8}

    \coordinate (@1) at #4 ;
    \coordinate (@2) at #5 ;

    \if #7v % Cotation verticale
        \coordinate (@0) at ($($#4!.5!#5$) + (#3,0)$) ; 
        \coordinate (@4) at (@0|-@1) ;
        \coordinate (@5) at (@0|-@2) ;
    \else
    \if #7h % Cotation horizontale
        \coordinate (@0) at ($($#4!.5!#5$) + (0,#3)$) ; 
        \coordinate (@4) at (@0-|@1) ;
        \coordinate (@5) at (@0-|@2) ;
    \else % cotation encoche
    \ifnum\pdfstrcmp{\unexpanded\expandafter{\@car#7\@nil}}{(}=\z@
        \coordinate (@5) at ($#7!#3!#5$) ;
        \coordinate (@4) at ($#7!#3!#4$) ;
    \else % cotation oblique    
        \coordinate (@5) at ($#5!#3!90:#4$) ;
        \coordinate (@4) at ($#4!#3!-90:#5$) ;
    \fi\fi\fi

    \draw[very thin,shorten >= 1.5pt,shorten <= -2*1.5pt] (@4) -- #4 ;
    \draw[very thin,shorten >= 1.5pt,shorten <= -2*1.5pt] (@5) -- #5 ;

    \IfBooleanTF #1 {% avec étoile
    \draw[Cote arrow,-] (@4) -- (@5);
    \coordinate (@99) at ($@4+|0.5*(@5-|@4)+#2$);
    \draw (@99)--(@99) node[Cote node,#8] {#6\strut};
    %\draw (@99)--(@99) node[Cote node,scale=#8,rotate=(#9)] {#6\strut};
    \draw[Cote arrow,<-] (@4) -- ($(@4)!-6pt!(@5)$) ;   
    \draw[Cote arrow,<-] (@5) -- ($(@5)!-6pt!(@4)$) ;   
    }{% sans étoile
    \ifnum\pdfstrcmp{\unexpanded\expandafter{\@car#7\@nil}}{(}=\z@
        \draw[Cote arrow] (@5) to[bend right] node[Cote node,#8] {#6\strut} (@4) ;
    \else
    \draw[Cote arrow] (@4) -- (@5) node[Cote node,#8] {#6\strut};
    \fi
    }}
    }
\makeatother



